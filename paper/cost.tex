\documentclass[conf]{new-aiaa}
%\documentclass[journal]{new-aiaa} for journal papers
\usepackage[utf8]{inputenc}

\usepackage{graphicx}
\usepackage{amsmath}
\usepackage[version=4]{mhchem}
\usepackage{siunitx}
\usepackage{longtable,tabularx}
\usepackage{footnote}
\usepackage{mhchem}
\usepackage{physics}
\usepackage{array,makecell,booktabs}
\newcolumntype{M}[1]{>{\centering\arraybackslash}m{#1}}
\usepackage[super]{nth}
\makesavenoteenv{tabular}
\setlength\LTleft{0pt}

\graphicspath{{figures/}}

\begin{document}

\section{Cost Model}
Introduce transcost, "top down", introduce CERs, element mass drives cost
performance model linked with cost model with element masses

\subsection{Cost Model Description}
(not-a-)pie chart - shows majority of cpf is first stage production
description of model components: dev, prod, ops
description of what's included in CPF and PPF (ie we are exluding dev cost)
quantification of uncertainty - CER data extraction/confidence bounds

\subsection{Validation of Cost Model}
validation violin plot

\bibliography{first_stage_recovery}

\end{document}