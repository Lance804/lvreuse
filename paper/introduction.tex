\documentclass[conf]{new-aiaa}
%\documentclass[journal]{new-aiaa} for journal papers
\usepackage[utf8]{inputenc}

\usepackage{graphicx}
\usepackage{amsmath}
\usepackage[version=4]{mhchem}
\usepackage{siunitx}
\usepackage{longtable,tabularx}
\usepackage{footnote}
\usepackage{mhchem}
\usepackage{physics}
\usepackage{array,makecell,booktabs}
\newcolumntype{M}[1]{>{\centering\arraybackslash}m{#1}}
\usepackage[super]{nth}
\makesavenoteenv{tabular}
\setlength\LTleft{0pt} 

\graphicspath{{figures/}}

\title{Strategies for Re-Use of Launch Vehicle First Stages}

\author{Matthew T. Vernacchia \footnote{Research Assistant, Department of Aeronautics and Astronautics, 77 Massachusetts Avenue, AIAA Student Member.}
and Kelly J. Mathesius  \footnote{Research Assistant, Department of Aeronautics and Astronautics, 77 Massachusetts Avenue, AIAA Student Member.}}
\affil{Massachusetts Institute of Technology, Cambridge, MA, 02139}


\begin{document}

\maketitle

\section{Introduction}

TODO
**Goal: make access to space easier by reducing costs and increasing launch rate

Few readers will doubt the importance of making access to space easier, by reducing launch costs and increasing launch rates. Starry-eyed explorers have long felt that humanity will not step out of its cradle unless the difficulty of doing so is reduced. Recently, more practical concerns have begun demanding cheaper and faster launch services. Many high-count commercial satellite constellations are under development for the Earth observation and telecommunications markets \cite{SIA2017, Henry2017}. Militaries have also indicated a growing interest in distributing their space capabilities across higher-count constellations that can be quickly replenished (TODO cite DARPA blackjack, etc.).

These trends in the satellite market mean that launch providers may need to reduce costs and increase launch rates to remain competitive. As serial production drops the cost of satellites, customers may become more sensitive to launch costs. And large constellations require high launch rates to be deployed in a viable timeframe.


**Reuse has been touted as a way to acheive this goal

Dating back to von Braun, there is a long history of rocket engineers proposing re-usability as a means to reduce costs and increase launch rate (TODO cite von Braun shuttle concept). The essential argument has been that the high costs of launch are driven by the difficulty of producing and testing rocket hardware. Re-use would enable this cost to be spread over many flights, thereby reducing the cost per flight. Some proposals have also argued that a re-usable vehicle could streamline launch operations, reducing operational costs as well (TODO cite DC-X).


**However, full-vehicle reuse does not seem to be ready for prime-time (yet)

However, full launch vehicle re-use does no
shuttle tried to reduce costs via reuse of whole vehicle and failed. Other whole reuse concepts (e.g. X-33, Delta Clipper) also failed.

**Partial reuse, on the other hand, is gaining traction
Recently, renewed interest in re-use, but focused only on first stage

**What this paper does
this paper will establish a common, first principles framework to evaluate the relative merit of proposed strategies to recover the first stage of a two-stage launch vehicle. 


\bibliography{first_stage_recovery}

\end{document}