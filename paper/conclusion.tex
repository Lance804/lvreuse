\documentclass[conf]{new-aiaa}
%\documentclass[journal]{new-aiaa} for journal papers
\usepackage[utf8]{inputenc}

\usepackage{graphicx}
\usepackage{amsmath}
\usepackage[version=4]{mhchem}
\usepackage{siunitx}
\usepackage{longtable,tabularx}
\usepackage{footnote}
\usepackage{mhchem}
\usepackage{physics}
\usepackage{array,makecell,booktabs}
\newcolumntype{M}[1]{>{\centering\arraybackslash}m{#1}}
\usepackage[super]{nth}
\makesavenoteenv{tabular}
\setlength\LTleft{0pt}

\graphicspath{{figures/}}

\begin{document}

\section{Conclusion}

takeaways from strategies to choose
downrange propulsive landing appears to be the winner
lots of uncertainty on winged vehicles
partial reuse strategies don't seem to cut down on production cost, won't enable as large of a launch rate increase as a fully reusable stage

takeaways for when reuse makes sense
makes more sense for larger vehicles where hardware costs dominate, makes less sense for small-sat launchers
performance penalty for reuse is less severe for lower energy reuses - reuse would make sense for large consellations going to LEO
launch rate and schedule are huge cost drivers - if reuse could drive up the launch rate and there is a market for more launches, operations costs will decrease, return on development cost investment would happen sooner
if schedule is important, there might be reason to prioritize launch site landings and reduce turn-around time

\bibliography{first_stage_recovery}

\end{document}