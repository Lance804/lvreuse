\documentclass[conf]{new-aiaa}
%\documentclass[journal]{new-aiaa} for journal papers
\usepackage[utf8]{inputenc}

\usepackage{graphicx}
\usepackage{amsmath}
\usepackage[version=4]{mhchem}
\usepackage{siunitx}
\usepackage{longtable,tabularx}
\usepackage{footnote}
\usepackage{mhchem}
\usepackage{physics}
\usepackage{array,makecell,booktabs}
\newcolumntype{M}[1]{>{\centering\arraybackslash}m{#1}}
\usepackage[super]{nth}
\makesavenoteenv{tabular}
\setlength\LTleft{0pt}

\graphicspath{{figures/}}

\begin{document}

\section{Conclusion}

This paper has introduced preliminary models for the performance and cost of launch vehicles employing various strategies for first stage reuse. We used Monte Carlo techniques to evaluate these models while taking into account uncertainty in the model input parameters.

Winged reuse strategies were found to have very uncertain and likely high costs, and partial reuse strategies did not offer meaningful cost savings. In contrast, downrange propulsive landing offers significant cost savings and is probably the dominant strategy from a cost per flight standpoint. It could plausibly reduce launch costs to $\frac{1}{2}$ to $\frac{1}{3}$ the cost of an expendable vehicle.

Reuse makes most sense for large launch vehicles with high launch rates. A small sizes, the first stage hardware costs are only a small fraction of the cost per flight, so reuse is not particularly helpful. High launch rates (>20/year) are likely required to pay off the development costs of reuse. However, first stage reuse may facilitate higher launch rates if first stage production is the rate limiting step. Further, interest in high-count LEO constellations indicates that the market demand may be sufficient to sustain these high launch rates. Thus, the present analysis indicates that first stage reuse is a viable route to reducing launch costs for medium to heavy lift launch vehicles.


\bibliography{first_stage_recovery}

\end{document}